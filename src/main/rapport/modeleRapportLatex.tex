% commande pour compiler: pdflatex modeleRapportLatex.tex -output-directory=output && mv output/modeleRapportLatex.pdf ./

\documentclass[a4paper]{article} % Utilisation de la classe article
% Options possibles : 10pt, 11pt, 12pt (taille de la fonte)
%                     oneside, twoside (recto simple, recto-verso)
%                     draft, final (stade de développement)

\usepackage[utf8]{inputenc}  % LaTeX, comprends les accents !
\usepackage[T1]{fontenc}  % Police contenant les caractères français
\usepackage[french]{babel}  % Le document est en français
\usepackage{fullpage}  % pour les marges
\usepackage{graphicx}  % pour inclure des images

%\pagestyle{headings}  % Pour mettre des entêtes avec les titres
						% des sections en haut de page

\usepackage{datetime}  % pour la date
\newdate{frontpagedate}{12}{05}{2024}  % jour, mois, année


\begin{document}
%\begin{center}               % pour centrer 
 % \includegraphics[scale=1]{}   % insertion d'une image
%\end{center}
\begin{titlepage}
\begin{center}
\vspace{2cm}
%\textsc{ Oregon State University}\\[1.5cm]
\includegraphics[width=0.4\textwidth]{root/UM1.png}~\\[1cm]
\vspace{2cm}

% Title
\hrule
\vspace{.5cm}
{\huge\bfseries{Hex-Ta(c)tique\\Projet de programmation 2}} % title of the report
\vspace{.5cm}

\hrule
\vspace{1.5cm}

\textsc{\textbf{Auteurs}}\\
\vspace{.5cm}
\centering

% add your name here
AL AYOUBI Ibrahim\\
BIGEY Raphaël\\
BONETTI Timothée\\
LEJEUNE Ivan\\


\vspace{1cm}

\textsc{\textbf{Encadrant}}\\
\vspace{.5cm}
\centering

% add your name here
DA-SILVA Sébastien

\vspace{4cm}

\centering \displaydate{frontpagedate} % Dags dato
\end{center}
\end{titlepage}  % page de garde

% \begin{abstract}     % Résumé du travail

%   \emph{Description très succinte du problème et des différentes étapes de réalisation}

% \end{abstract}
\newpage

\tableofcontents  % Table des matières

\newpage
\section{Présentation du sujet}


% présenter le problème étudié et le contexte dans lequel il se positionne
Dans le cadre de notre projet de programmation de fin de semestre, nous avons 
choisi de travailler sur l'implémentation et l'étude de la résolution de jeux 
de combinatoire.

Le projet consiste d'abord à créer une version informatique du jeu du hex, un 
jeu de stratégie combinatoire abstrait inventé par Piet Hein et John Nash, puis 
reporter nos efforts sur le jeu de l'Awalé, un jeu de stratégie combinatoire 
abstrait d'origine africaine.

% motiver l'intérêt du problème étudié par rapport à votre parcours d'études
% et au monde de l'informatique
Ce projet est motivé par notre intérêt pour la théorie des jeux, et par notre 
désir de comprendre les mécanismes de résolution de jeux de stratégie 
combinatoire.
L'étude de ces jeux nous permettra de mieux comprendre les algorithmes de 
recherche et d'optimisation, et de nous familiariser avec les techniques de 
programmation avancée.
A RAJOUTER : L'interet de la résolution de jeux de combinatoire par rapport a 
nos etudes et au monde de l'informatique

% présenter les différentes approches possibles pour la résolution du problème
% et en particulier celle choisie
Pour résoudre ces jeux, plusieurs approches sont possibles.
La première chose à prendre en compte est le choix de la représentation du jeu,
et la manière dont les règles du jeu seront implémentées.

Ensuite, il faudra choisir une méthode de résolution du jeu.
Pour le jeu du hex, nous avons choisi d'implémenter un algorithme de recherche 
minimax avec élagage alpha-bêta, et pour le jeu de l'Awalé, nous avons choisi 
d'implémenter un algorithme de recherche Dijkstra.

D'autres approches sont possibles, comme l'implémentation d'un algorithme de 
Monte Carlo Tree Search, ou l'utilisation de réseaux de neurones pour apprendre 
à jouer au jeu.
A RAJOUTER : Les différentes approches possibles pour la résolution de jeux de 
combinatoire

% donner le cahier des charges détaillé :
Le cahier des charges détaillé est disponible en annexe.
A RAJOUTER : Le cahier des charges détaillé

\section{Technologies utilisées}

% présenter les langages de programmation et les outils utilisés dans le cadre
% du projet
Pour la réalisation de ce projet, nous avons utilisé les langages de programmation suivants :
\begin{itemize}
	\item Python pour l'implémentation des algorithmes de résolution des jeux et la logique du jeu,
	\item HTML, CSS et JavaScript pour l'implémentation de l'interface graphique des jeux,
	\item UML pour la modélisation des classes et des cas d'utilisation,
	\item Git pour la gestion du code source et le suivi des versions,
	\item Visual Studio Code pour l'écriture du code et le débogage,
	\item GitHub pour l'hébergement du code source et la collaboration,
	\item LaTeX pour la rédaction du rapport,
	\item Draw.io pour la création des diagrammes UML.
\end{itemize}

% justifier le choix et l'intérêt des langages utilisés

\section{Développements Logiciel : Conception, Modélisation, Implémentation} 

% présenter les développements logiciel réalisés dans le cadre du projet


% présenter les principaux modules du logiciel développé dans le cadre du projet
% Utiliser le langage UML pour la modélisation : donner le diagramme de cas
% d'utilisation et le diagramme des classes


% décrire les fonctionnalités de l'inteface graphique implémentée (si votre 
% logiciel dispose d'une interface graphique)


% présenter les principales structures de données définies dans le cadre du
% projet. Décrire le format des données en entrée ou encore les conventions
% utilisées pour les entrées de vos programmes. Décrire les procédures de
% lecture et validation des entrées.


% Statistiques : nombre de modules/composantes/classes/scripts développés.
% Nombre de lignes de code.

\section{Algorithmes et Analyse}


% présenter les principaux algorithmes implémentés. En illustrer le
% fonctionnement avec des exemples. Attention : il s'agit ici de choisir 1 ou 2
% algorithmes intéressants, et non pas de présenter tous les algorithmes
% implémentés.


% evaluer la complexité théorique en temps des algorithmes présentés.


\section{Analyse des résultats}

% illustrer les performances ainsi que l'efficacité du logiciel implémenté
% a l'aide de graphiques.


% analyser (et comparer, si plusieurs) les performances des solutions 
% implémentées


% présenter les bancs d'essais (ou les procédures utilisées pour la génération)
% des données) utilisés pour les tests du logiciel.

\section{Gestion du Projet}

% présenter la gestion du projet et les documents de planification éventuellement
% rédigés (par exemple, le diagramme de Gantt).


% discuter les changements majeurs effectués en cours de projet

\section{Bilan et Conclusions}

% indiquer les fonctionnalités mise en oeuvre par rapport au cahier des charges
% de départ, les points ouverts et les perspectives pour le projet.

\section{Bibliographie}

% indiquer les ressources bibliographiques utilisées pour le projet


% donner des références exactes des travaux cités dans le texte permettant de
% trouver les articles ou livres cités en bibliothèque ou sur internet.

\section{Annexes}

% par exemple: fragments de code, manuel d'utilisation du logiciel etc


% \begin{thebibliography}{9}
% \bibitem{texbook}
% Donald E. Knuth (1986) \emph{The \TeX{} Book}, Addison-Wesley Professional.

% \bibitem{lamport94}
% Leslie Lamport (1994) \emph{\LaTeX: a document preparation system}, Addison
% Wesley, Massachusetts, 2nd ed.
% \end{thebibliography}



\end{document}

