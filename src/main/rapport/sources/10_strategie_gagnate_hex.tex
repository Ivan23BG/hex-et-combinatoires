\section{HexGame et strategie gagnante}

Au Hex, pour toutes les tailles de plateaux il existe une strategie gagnante theorique pour le joueur qui commence:
Celle ci n'est pas connue pour la plupart des tailles de plateaux car elle demande une conaissance totale de toutes 
les parties possible ce qui n'est pas calculable en temps realiste.

\paragraph{Stratégies gagnantes et arbre du jeu}
Modelisons le jeu de Hex a l'aide d'un arbre representant toutes les parties possibles, L'arbre commence avec la position 
initiale, un plateau vide. De cette racine partent autant de branches qu'il y a de possibilités pour 
le premier coup du premier joueur. Et ainsi de suite jusqu'a tomber sur des positions gagnantes pour l'un ou l'autre joueur
(exemple ci dessou pour un plateau de 2*2)
%faire exemple
\includegraphics[width=0.4\textwidth]{root/strategie_gagnante.png}

L'arbre va nous aider à nous convaincre qu'il y a bien une stratégie gagnante pour l'un ou l'autre
des deux joueurs. Le principe consiste à colorier tous les nœuds de l'arbre en blanc ou en noir, chaque
nœud colorié correspondant à une position à partir de laquelle le joueur de la couleur correspondante
possède une stratégie gagnante. 
Nous commençons par colorier en blanc les feuilles représentant une fin de
partie gagnée par les blancs, en noir celles correspondant à la victoire du joueur noir.
Le reste du coloriage se fait progressivement. À un moment donné, on veut attribuer une couleur
à un nœud qui n'en a pas encore, mais dont toutes les branches descendantes mènent à des nœuds
déjà coloriés. Supposons que ce nœud représente une position à partir de laquelle c'est aux blancs de
jouer. Si l'une au moins des branches issues du nœud mène à un nœud blanc, alors on colorie le nœud
en blanc. Dans le cas contraire, c'est-à-dire si toutes les branches mènent à des nœuds coloriés en
noir, on le colorie en noir. On procède de façon symétrique si c'est aux noirs de jouer.
De cette manière nous sommes assurés de remplir l'abre en entier de nos couleurs jusqu'a la racine
qui donc possèdera une strategie gagnante.


\paragraph {remarque:}
On voit que pour un plateau de 2*2 l'arbre de tous les possibles possède deja 24 feuilles et 52 branches
il est calculé que pour le Hex a 11*11 la Game-tree complexity ou le nombre de feuilles de l'arbre est
approximativement 10^98 et le nombre de positions totales que peut prendre le jeu est de 2.4*10^56 contre
4.2*10^56
