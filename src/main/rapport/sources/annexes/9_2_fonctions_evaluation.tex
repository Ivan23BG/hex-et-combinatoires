\subsection{Fonctions d'évaluation}

L'évaluation d'une position est une étape cruciale dans la conception d'un programme de jeu. 
Elle permet de déterminer la force relative des positions des deux joueurs et d'orienter le choix du coup à jouer à la fois
pour une IA et pour un joueur humain qui conceptualise la stratégie à suivre.
Dans le cadre de notre projet, nous avons implémenté plusieurs fonctions d'évaluation pour le jeu Hex.

\subsubsection{Fonction d'évaluation initiale}
La première fonction d'évaluation implémentée était très simple. Elle attribuait un score de 1 à chaque case du plateau
occupée par une pièce du joueur et un score de -1 à chaque case occupée par une pièce de l'adversaire.
Le score final de la position dépendait alors de la différence entre le nombre de cases occupées par le joueur
et le nombre de cases occupées par l'adversaire autour de la case centrale du plateau.
Cette fonction d'évaluation a été implémentée pour tester le fonctionnement du jeu et de l'IA, mais elle s'est avérée
trop simpliste pour être efficace.

\subsubsection{Fonction d'évaluation basée sur la proximité au bord}
La deuxième fonction d'évaluation implémentée prenait en compte la proximité des cases au bord du plateau.
Elle attribuait un score plus élevé aux cases proches du bord, car elles sont plus stratégiques et plus difficiles à bloquer.
Cette fonction d'évaluation a permis d'améliorer les performances de l'IA, mais elle n'était pas suffisante pour
prendre en compte les stratégies de blocage et de connexion des pièces.

De plus, lors de l'implémentation de cette fonction, nous avons rencontré des difficultés pour catégoriser les cases
qui étaient similaires en termes de proximité au bord. Cela nous a conduit à des situations où l'IA plaçait ses pièces
dans un ordre aléatoire, ce qui n'était pas optimal.

\subsubsection{Fonction d'évaluation basée sur les voisins}
La troisième fonction d'évaluation implémentée comptait le nombre de voisins de chaque joueur et essayait de bloquer
l'ajout d'un autre voisin pour ne pas former une chaîne en plus. En bref, l'idée de cette fonction était de vérifier
l'intérêt à bloquer les chaînes de l'adversaire plutôt que de former les siennes.
Cette fonction d'évaluation nous a permis de mieux comprendre les stratégies de blocage et de connexion des pièces,
mais elle n'était toujours pas suffisante et était facile à contrer.

\subsubsection{Fonction d'évaluation basée sur la position actuelle}
La quatrième fonction d'évaluation implémentée évaluait la position actuelle sur le plateau de jeu pour déterminer
la force relative des positions des deux joueurs. Elle prenait en compte les composantes connectées des pions de chaque joueur,
leur proximité au centre du plateau et la possibilité de victoire imminente.
C'était un grand pas en avant par rapport aux fonctions d'évaluation précédentes car elle changeait activement
de stratégie en fonction de la position actuelle sur le plateau. Elle était donc moins `prévisible' pour l'adversaire et 
moins facile à contrer.
Néanmoins, cette fonction d'évaluation était toujours basée sur des critères statiques et ne prenait pas complètement
en compte les stratégies de blocage et de connexion des pièces.

\subsubsection{Fonction d'évaluation basée sur l'algorithme de Dijkstra}
Un des gros problèmes rencontrés lors de l'implémentation des fonctions d'évaluation était qu'on implémentait
des `idées' de stratégies sans vraiment les formaliser. Cela nous a conduit à des fonctions d'évaluation très complexes
et difficiles à optimiser sans pour autant être efficaces.
On a donc décidé d'implémenter une fonction d'évaluation basée sur l'algorithme de \emph{Dijkstra} pour mettre à jour
les scores des cases sur le plateau en trouvant les chemins les plus courts du point de départ du joueur vers chaque case.
Cette fonction d'évaluation permettait de déterminer les cases qui rapprochaient le joueur de la victoire et de les
prioriser dans le choix du coup à jouer.

On avait alors une fonction d'évaluation qui `visualisait' les chemins possibles pour chaque joueur et qui permettait
de déterminer les coups les plus stratégiques à jouer. 
Cependant, cette fonction d'évaluation était très complexe (en termes de temps de calcul) car elle nécessitait de parcourir
une grande partie du plateau pour chaque coup. Cela a conduit à des temps de calcul très longs et à des performances
globales de l'IA qui n'étaient pas satisfaisantes.

\subsubsection{Le problème de la complexité}
La dernière fonction d'évaluation implémentée, bien que plus efficace que les précédentes, souffrait toujours du problème
de complexité. En effet, les fonctions d'évaluation étaient de plus en plus complexes et nécessitaient de plus en plus de
calculs pour déterminer le score d'une position. Plusieurs choix se présentent alors pour résoudre ce problème:
\begin{itemize}
    \item Réduire la complexité des fonctions d'évaluation en simplifiant les critères pris en compte.
    \item Optimiser les fonctions d'évaluation pour réduire le temps de calcul.
    \item Implémenter des algorithmes de recherche plus efficaces pour déterminer le coup à jouer.
\end{itemize}
Dans le cadre de notre projet, nous avons choisi de simplifier les fonctions d'évaluation et d'optimiser les algorithmes
de recherche pour améliorer les performances de l'IA.\@
Cela a permis de réduire la complexité du programme et d'obtenir des résultats plus satisfaisants en termes de temps de 
calcul et de performances globales de l'IA.\@
Mais, comme pour tout problème de recherche, il n'y a pas de solution parfaite et il est important 
de trouver un compromis entre la complexité des fonctions d'évaluation et les performances de l'IA:\@
\begin{itemize}
    \item Réduire la complexité des fonctions d'évaluation peut entraîner une perte de précision dans l'évaluation des positions
    et une moins bonne performance de l'IA.\@
    \item Optimiser les fonctions d'évaluation peut permettre d'améliorer les performances de l'IA, mais cela nécessite
    beaucoup de travail et atteint rapidement des limites: il est difficile d'optimiser une fonction d'évaluation
    complexe sans la simplifier.\@
    \item Implémenter des algorithmes de recherche plus efficaces peut permettre d'améliorer les performances de l'IA,
    mais cela nécessite également beaucoup de travail et de recherche pour trouver les algorithmes les plus adaptés
    au jeu et à l'IA.\@
\end{itemize}

\subsubsection{Conclusion}
En conclusion, cette partie aborde les différentes approches d'évaluation implémentées dans le projet.
Les évaluations ont évolué pour être plus précises et tenir compte de la proximité du bord, des stratégies de blocage
et de la recherche de chemins optimaux vers la victoire, tout en essayant de réduire le temps de calcul et la complexité
du programme.
Cependant, le problème de la complexité reste un défi majeur dans la conception d'une IA pour le jeu Hex, et il est donc
toujours possible d'améliorer les fonctions d'évaluation et les algorithmes de recherche pour obtenir de meilleurs résultats.
Dans le cadre de notre projet, nous avons choisi de passer ce temps à améliorer d'autres aspects du jeu,
mais il reste encore beaucoup à faire pour obtenir une IA optimale pour le jeu Hex.

