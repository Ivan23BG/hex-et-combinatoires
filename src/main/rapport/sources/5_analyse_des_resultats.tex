% illustrer les performances ainsi que l'efficacité du logiciel implémenté
% a l'aide de graphiques.


% analyser (et comparer, si plusieurs) les performances des solutions 
% implémentées


% présenter les bancs d'essais (ou les procédures utilisées pour la génération)
% des données) utilisés pour les tests du logiciel.

\section{Algo MinMax observations}
une fois l'implémentation de l'algorithme MinMax faite sur nos deux
jeux combinatoires nous pouvons nous demander est ce que cet algorithme est adapté pour le jeu de hex et le awale ?
Nous observons que les performances du MinMax sont très différentes d'un jeu a l'autre, étudions pourquoi:

\subsection {Hexgame : Mauvaise performance}
l'implémentation de l'algo MinMax sur le jeu de Hex n'arrive quasiment jamais a battre un humain et cela pour 
2 raisons principales: La faible profondeur et la difficulté de trouver une fonction d'évaluation satisfaisante.
\paragraph {Trop grande complexité} Le premier Problème provient du nombre de coup possible a chaque tour: 
en effet par exemple sur un plateau classique de 13 par 13 le premier joueur a 169 coups possibles et au coup suivant
il y en a 168 etc\dots cela rend les calculs lents car par exemple: si il joue en premier 
et que la profondeur demandée est de 6 alors l'ordinateur doit calculer 2.1298467e+13 (ou 169*168*167*166*165*164) 
fois la fonction d'évaluation ce qui n'est pas réalisable en temps réaliste. Avec une profondeur seulement de 
4 comme nous l'avons implémente sur un plateau de 11*11 ou plus l'ordinateur prend déjà plusieurs 
secondes a calculer le meilleur coup. Notons que ces calculs ne prennenet pas en compte l'élagage alpha beta
qui optimise significativement le min max. Mais meme si une portion des noeuds n'est pas calculés le calcul reste trop lent
on peut noter aussi que la performance de l'elagage alpha beta depend de la qualité de la fonction d'evaluation qui comme nous allons
le voir n'est pas assurée.

\paragraph{Fonction d'evaluation : casse tête}Le MinMax n'arrivant quasiment jamais a une feuille terminale la fonction
d'évaluation du jeu de Hex devait être performante et de faible cout en temps. Or évaluer si une position est gagnante 
ou non s'est révélé très difficile et couteux. Et cela en grande partie a cause du fait que le hex est un jeu qui 
possède de nombreuses stratégies, par exemple au début de la partie il est préférable de jouer au millieu du plateau 
pour se laisser le maximum de possibilités mais plus tard dans la partie jouer au centre n'est pas forcement 
une bonne idée. Un autre aspect fondamental du jeu compliqué a faire comprendre a un ordinateur ce sont les ponts 
(cf exemple) dans ces cas la le joueur est assurer de pouvoir connecter 2 pièces a la condition que l'autre joueur ne 
joue pas a cet endroit. Il faut donc ne pas jouer a cet endroit pour jouer autre part mais si l'adversaire joue a cet 
endroit il faut absolument compléter le pont. On peut noter que quand la profondeur est plus grande le MinMax arrive 
souvent a voir les ponts mais on en revient au problème n°1: la complexité.
Notre fonction d'évaluation finale essaye de créer les groupes de hex les plus grands en largeur ou en hauteur en 
fonction du joueur elle bat parfois des humains (surtout sur des petits plateaux) mais prend du temps et des que le 
taille du plateau est trop grande le problème de complexité devient ingérable
%faire exemple de pont


\subsection {Awale : Bonne performance}
\paragraph {tout le contraire du hex:}En effet au Awale le nombre maximum de coup possible est toujours de 6 cela 
réduit grandement la complexité nous pouvons mettre un profondeur de 10 par exemple sans craindre un trop long temps 
de calcul cela rend l'algo bien meilleur dans l'ensemble.
De plus une fonction d'évaluation naturelle apparait avec les règles: il suffit que le joueur maximisant maximise 
ses points et minimise ceux de sont adversaire pour gagner. Chose que fait très bien l'algorithme, En effet avec 
une profondeur de 8 et cette fonction simple et très rapide nous n'arrivons pas a battre l'ordinateur.

\paragraph {Conclusion}En conclusion nous pouvons dire que l'algo MinMax fonctionne très bien quand la profondeur 
maximale est grande et que le jeu n'est pas trop complexe a comprendre pour un ordinateur. 
Notons que dans le cas du Hex le principal problème est celui du temps de calcul si nous pouvions mettre une grande 
profondeur maximale même avec une fonction d'évaluation relativement simple l'algorithme jouerai bien.