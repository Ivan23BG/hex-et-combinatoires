\section{Présentation du sujet}
% présenter le problème étudié et le contexte dans lequel il se positionne
Les jeux de société ont toujours été un moyen de divertissement populaire, mais au-delà 
de leur aspect récréatif, ils constituent également des domaines d'étude intéressants pour les 
chercheurs en informatique, en mathématiques et en intelligence artificielle. 
Dans ce contexte, et dans le cadre de notre projet de programmation de L3 Informatique, 
nous avons choisi de nous intéresser à la théorie des jeux combinatoires. Nous avons ainsi mis 
en œuvre les jeux de stratégie combinatoire abstrait du Hex et de l'Awalé, reconnaissant ainsi 
leur valeur non seulement comme des défis ludiques, mais aussi comme des sujets de recherche 
passionnants pour explorer les interactions stratégiques et algorithmiques entre les joueurs.

Avant de poursuivre notre discussion sur notre motivation à étudier ces jeux, il est pertinent
de clairifier ce que l'on entend par «jeux de stratégie combinatoire abstrait». Ce sont des jeux 
(généralement de société) tel que:
\begin{itemize}
	\item opposant généralement deux joueurs ou deux équipes (ou bien un joueur humain seul 
	contre un ordinateur « intelligent »)
	\item dans lequel les joueurs ou équipes jouent à tour de rôle.
	\item dont tous les éléments sont connus (jeu à information complète).
	\item où le hasard n'intervient pas pendant le déroulement du jeu.
\end{itemize}
En d'autres termes, dans les jeux de stratégie combinatoire, la victoire dépend entièrement 
des actions des joueurs et de leur capacité à anticiper et contrer les mouvements de l'adversaire. 


% motiver l'intérêt du problème étudié par rapport à votre parcours d'études
% et au monde de l'informatique
L'étude de ces jeux nous motive à mieux comprendre les algorithmes de 
recherche et d'optimisation, et de nous familiariser avec les techniques de 
programmation avancée. A l'avenir, ces résultats pourront être utilisés dans
un cadre plus général pour la résolution de problèmes plus complexes, par exemple
pour les échecs ou le go. 


% présenter les différentes approches possibles pour la résolution du problème
% et en particulier celle choisie
Il existe plusieurs approches possibles pour la résolution de jeux de stratégie
combinatoire. Parmi les approches les plus courantes, on trouve les algorithmes
de recherche en profondeur, les algorithmes de recherche de chemin, les
algorithmes de recherche de meilleure réponse, les algorithmes de Monte-Carlo,
les algorithmes de renforcement, etc.
Pour ce projet, nous avons choisi d'implémenter deux algorithmes en particulier
pour la résolution des jeux de hex et d'Awalé: l'algorithme Minimax avec élagage
alpha-bêta pour le jeu de hex, et l'algorithme de Dijsktra pour le jeu de l'Awalé.

Les principaux avantages de ces algorithmes sont les suivants:
\begin{itemize}
	\item L'algorithme Minimax avec élagage alpha-bêta est un algorithme de recherche
	qui permet de trouver la meilleure stratégie pour un joueur dans un jeu à deux
	joueurs. Cet algorithme est très efficace pour les jeux de stratégie combinatoire
	comme le hex, car il permet de réduire le nombre de nœuds explorés lors de la
	recherche de la meilleure stratégie.
	\item L'algorithme de Dijsktra est un algorithme de recherche de chemin qui permet
	de trouver le chemin le plus court entre deux nœuds dans un graphe. Cet algorithme
	est très efficace pour le jeu de l'Awalé, car il permet de trouver la meilleure
	stratégie pour un joueur en minimisant le nombre de graines capturées par l'adversaire.
\end{itemize}

% donner le cahier des charges détaillé :
Le cahier des charges détaillé est disponible en annexe.
A RAJOUTER\@: Le cahier des charges détaillé