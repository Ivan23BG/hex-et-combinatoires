\section{Présentation du sujet}
% présenter le problème étudié et le contexte dans lequel il se positionne
Dans le cadre de notre projet de programmation de L3 Informatique, nous avons
choisi de travailler sur la résolution de jeux de stratégie combinatoire.
Nous avons choisi de nous intéresser à deux jeux de stratégie combinatoire en
particulier: le jeu du hex et le jeu de l'Awalé.

% motiver l'intérêt du problème étudié par rapport à votre parcours d'études
% et au monde de l'informatique
Ce projet est motivé par notre intérêt pour la théorie des jeux, et par notre 
désir de comprendre les mécanismes de résolution de jeux de stratégie 
combinatoire.
L'étude de ces jeux nous permettra de mieux comprendre les algorithmes de 
recherche et d'optimisation, et de nous familiariser avec les techniques de 
programmation avancée. A l'avenir, ces résultats pourront être utilisés dans
un cadre plus général pour la résolution de problèmes complexes, par exemple
pour les échecs ou le go.

% présenter les différentes approches possibles pour la résolution du problème
% et en particulier celle choisie
Il existe plusieurs approches possibles pour la résolution de jeux de stratégie
combinatoire. Parmi les approches les plus courantes, on trouve les algorithmes
de recherche en profondeur, les algorithmes de recherche de chemin, les
algorithmes de recherche de meilleure réponse, les algorithmes de Monte-Carlo,
les algorithmes de renforcement, etc.
Pour ce projet, nous avons choisi d'implémenter deux algorithmes en particulier
pour la résolution des jeux de hex et d'Awalé: l'algorithme Minimax avec élagage
alpha-bêta pour le jeu de hex, et l'algorithme de Dijsktra pour le jeu de l'Awalé.

Les principaux avantages de ces algorithmes sont les suivants:
\begin{itemize}
	\item L'algorithme Minimax avec élagage alpha-bêta est un algorithme de recherche
	qui permet de trouver la meilleure stratégie pour un joueur dans un jeu à deux
	joueurs. Cet algorithme est très efficace pour les jeux de stratégie combinatoire
	comme le hex, car il permet de réduire le nombre de nœuds explorés lors de la
	recherche de la meilleure stratégie.
	\item L'algorithme de Dijsktra est un algorithme de recherche de chemin qui permet
	de trouver le chemin le plus court entre deux nœuds dans un graphe. Cet algorithme
	est très efficace pour le jeu de l'Awalé, car il permet de trouver la meilleure
	stratégie pour un joueur en minimisant le nombre de graines capturées par l'adversaire.
\end{itemize}

% donner le cahier des charges détaillé :
Le cahier des charges détaillé est disponible en annexe.
A RAJOUTER\@: Le cahier des charges détaillé