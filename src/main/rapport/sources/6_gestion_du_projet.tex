\section{Gestion du Projet}

% présenter la gestion du projet et les documents de planification éventuellement
% rédigés (par exemple, le diagramme de Gantt).
Pour la gestion du projet, nous avons suivi un cycle de développement similaire à celui de la méthode Scrum. 
Nous avons donc défini des sprints courts de quelques jours chaque semaine, avec une réunion hebdomadaire
le mercredi pour faire le point sur l'avancement du projet. 
Ces réunions étaient l'occasion de discuter des tâches effectuées, de celles à venir, et de résoudre les problèmes rencontrés.
Nous avons également utilisé le logiciel Jira pour gérer les tâches et les sprints, et pour suivre l'avancement du projet.

\subsection{Diagramme de Gantt}

Officiellement, nous n'avons pas utilisé de diagramme de Gantt pour la gestion du projet, mais nous avons tout de même
réalisé un planning prévisionnel des tâches à effectuer.
Ce planning a été réalisé en début de projet, et a été mis à jour régulièrement pour refléter l'avancement du projet.
Il a été utilisé pour définir les tâches à effectuer pour chaque sprint, et pour suivre l'avancement du projet.
Un exemple de diagramme de Gantt est présenté en annexe~\ref{fig:diagramme_de_gantt}.


% discuter les changements majeurs effectués en cours de projet
\subsection{Changements Majeurs}

Comme mentionné précédemment, nous avons effectué des changements majeurs en cours de projet.
Ces changements ont été discutés en réunion, et ont été validés par l'ensemble de l'équipe.
Ils ont été intégrés au planning prévisionnel, et ont été pris en compte dans la gestion du projet.
Les changements majeurs effectués en cours de projet sont les suivants:

\begin{itemize}
    \item Changement de la hiérarchie des classes: nous avons changé la hiérarchie des classes pour
    une homogénéisation du code et une meilleure compréhension.
    \item Changement de l'organisation des fichiers: nous avons changé l'organisation des fichiers pour
    mieux organiser le code et faciliter la maintenance. Cela a initialement été fait pour le backend, puis
    pour le frontend. Beaucoup de problèmes ont été rencontrés lors de ces changements, mais ils ont permis
    d'améliorer la qualité du code.
    \item Changement de la gestion des erreurs: nous avons changé la gestion des erreurs pour une meilleure
    gestion des exceptions et une meilleure gestion des erreurs. Cela a permis d'améliorer la robustesse du code.
    Cela a également renforcé la sécurité du code sans nuire à la performance ou au debuggage.
\end{itemize}
