% Source File: src/main/rapport/sources/2_technologies_utilisees.tex
\section{Technologies utilisées}

% présenter les langages de programmation et les outils utilisés dans le cadre
% du projet
Pour la réalisation de ce projet, nous avons utilisé les langages de programmation suivants:
\begin{itemize}
	\item Python pour l'implémentation des algorithmes de résolution des jeux et la logique des jeux,
	\item HTML, CSS et JavaScript pour l'implémentation de l'interface graphique des jeux,
	\item Git pour la gestion du code source et le suivi des versions,
	\item Visual Studio Code pour l'écriture du code et le débogage,
	\item GitHub pour l'hébergement du code source et la collaboration,
	\item LaTeX pour la rédaction du rapport,
	\item Lucidchart pour la création des diagrammes UML.\@
\end{itemize}

% justifier le choix et l'intérêt des langages utilisés
Nous avons choisi Python pour l'implémentation des algorithmes de résolution des 
jeux et la logique du jeu, car c'est un langage de programmation populaire et 
très puissant. Il offre de nombreuses bibliothèques et modules pour le 
développement d'applications complexes. 
Python est également un langage de programmation très simple et lisible, 
ce qui facilite la compréhension du code et la collaboration entre les membres de 
l'équipe.
À noter que Python est un langage relativement lent et que l'utilisation d'un autre
langage pour le backend pourrait réduire les temps de calcul, mais pour un projet de cette envergure,
cela ne nous a pas semblé très pertinent. De plus, nous étions déjà familiers avec Python
et nous avons préféré continuer à l'utiliser pour ce projet.

Nous avons choisi HTML, CSS et JavaScript pour l'implémentation de l'interface
graphique des jeux. Ce sont des langages de programmation efficaces et 
simples à comprendre. Ils permettent de créer des interfaces graphiques intéractives 
et ergonomiques. HTML est un langage de balisage qui permet de structurer les pages
web. CSS est un langage de style qui permet de mettre en forme les pages web, et
JavaScript est un langage de programmation qui permet de rendre les pages web
intéractives.

Nous avons choisi Git et GitHub pour la gestion du code source et le suivi des
versions car ce sont des outils puissants avec lesquels nous sommes très
familiers. Quoique Git soit un outil complexe, il permet une gestion efficace
du code source et facilite la collaboration entre les membres de l'équipe. GitHub
est une plateforme d'hébergement de code source qui permet de stocker, de gérer
et de partager le code source de manière sécurisée.

LaTeX a été choisi pour la rédaction du rapport. C'est un langage de
composition de documents très puissant et flexible, qui permet de créer des
documents de grande qualité typographique.
Étant donné que nous avions déjà utilisé LaTeX pour d'autres projets, nous avons
préféré continuer à l'utiliser pour ce projet.

Lucidchart a été choisi pour la création des diagrammes UML car c'est un outil assez
simple qui répond parfaitement à nos besoins. Il permet de créer des diagrammes
UML et de les exporter dans différents formats.

