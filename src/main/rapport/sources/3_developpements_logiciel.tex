\section{Développements Logiciel: Conception, Modélisation, Implémentation} 

% présenter les développements logiciel réalisés dans le cadre du projet
Pour la réalisation de ce projet, nous avons développé un logiciel qui permet de jouer
aux jeux de stratégie combinatoire abstraits du Hex et de l'Awalé. Ce logiciel est
composé de plusieurs modules, dont les principaux sont les suivants:

% présenter les principaux modules du logiciel développé dans le cadre du projet
% Utiliser le langage UML pour la modélisation : donner le diagramme de cas
% d'utilisation et le diagramme des classes
\begin{itemize}
    \item \textbf{Module Hex:} Ce module contient les classes et les fonctions nécessaires
    pour jouer au jeu de Hex. Il contient notamment la classe \texttt{HexBoard} qui 
    représente le plateau de jeu et les joueurs du jeu de Hex. Ce module contient également 
    les fonctions pour l'implémentation de l'algorithme Minimax avec élagage alpha-bêta pour 
    la résolution du jeu de Hex.
    
    \item \textbf{Module Awalé:} Ce module contient les classes et les fonctions nécessaires
    pour jouer au jeu de l'Awalé. Il contient la classe \texttt{AwaleBoard} qui représente le
    plateau de jeu et les joueurs du jeu de l'Awalé. Ce module contient également les fonctions
    pour l'implémentation de l'algorithme de Dijsktra pour la résolution du jeu de l'Awalé.
    
    \item \textbf{Module Interface Graphique:} Ce module contient les fichiers HTML, CSS et
    JavaScript nécessaires pour l'implémentation de l'interface graphique des jeux de Hex et
    d'Awalé. Il contient notamment les fichiers \texttt{home.html}, \texttt{hex.html} et
    \texttt{awale.html} qui permettent à l'utilisateur de choisir le jeu auquel il veut jouer
    et les paramètres de la partie. Il contient également des fichiers JavaScript pour la
    gestion des événements et des interactions avec l'utilisateur.
    
    \item \textbf{Module Tests:} Ce module contient les fichiers de tests unitaires pour les
    classes et les fonctions des modules Hex et Awalé. Il contient notamment les fichiers
    \texttt{test\_hex.py} et \texttt{test\_awale.py}, qui permettent de tester les classes
    et les fonctions des modules Hex et Awalé. ***TODO: Ajouter les tests unitaires***
\end{itemize}


% décrire les fonctionnalités de l'interface graphique implémentée (si votre 
% logiciel dispose d'une interface graphique)
L'interface graphique de notre logiciel comprend une page d'accueil qui permet à l'utilisateur
de choisir le jeu auquel il veut jouer (Hex ou Awalé), une page principale pour chaque jeu
qui permet à l'utilisateur de choisir les paramètres de la partie (taille du plateau (pour Hex),
mode de jeu (joueur contre joueur, joueur contre ordinateur, ordinateur contre ordinateur),
niveau de difficulté (pour l'ordinateur), etc.), et une page de jeu qui permet à l'utilisateur
de jouer au jeu choisi. L'interface graphique est implémentée en HTML, CSS et JavaScript, et
elle communique avec les modules Hex et Awalé via des appels de fonctions JavaScript à des
fonctions Python via des requêtes json et le module Flask.

% présenter les principales structures de données définies dans le cadre du
% projet. Décrire le format des données en entrée ou encore les conventions
% utilisées pour les entrées de vos programmes. Décrire les procédures de
% lecture et validation des entrées.
Les principales structures de données définies dans le cadre de ce projet sont les classes
\texttt{HexBoard} et \texttt{AwaleBoard} qui représentent les plateaux de jeu du Hex et de
l'Awalé, respectivement. Ces classes contiennent les attributs et les méthodes nécessaires
pour représenter les plateaux de jeu et les joueurs, et pour effectuer les opérations de jeu
(placement de pions, déplacement de pions, etc.). Les données en entrée des programmes sont
sous forme de chaînes de caractères qui représentent les paramètres de la partie (taille du
plateau, mode de jeu, niveau de difficulté, etc.). Les procédures de lecture et de validation
des entrées sont effectuées par les fonctions des modules Hex et Awalé qui vérifient que les
paramètres de la partie sont valides avant de commencer la partie.


% Statistiques : nombre de modules/composantes/classes/scripts développés.
% Nombre de lignes de code.
Voici quelques statistiques sur le logiciel développé dans le cadre de ce projet:
\begin{itemize}
    \item Nombre de modules (sans les tests): 39
    \item Nombre de classes: 11 (principalement dans les modules Hex et Awalé)
    \item Nombre de fonctions: 68 (en Python) et 99 (en JavaScript) pour 167 en tout
    \item Nombre de lignes de code: 1405 (en Python), 2008 (en JavaScript), 553 (en HTML), 1331 (en CSS)
    et 480 (en \LaTeX) pour un total de 5777 lignes de code
\end{itemize}