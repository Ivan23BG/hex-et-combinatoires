\section{Moteur de jeux combinatoires abstraits et algorithmes}


% présenter les principaux algorithmes utilisés comme moteur de jeu combinatoires absrtaits
%methode de monte carlo
%min max


% evaluer la complexité théorique en temps des algorithmes présentés.

Implémenter un algorithme pour jouer à des jeux combinatoires contre l'ordinateur est l'un des objectifs principaux du projet. 
Néanmoins, plusieurs questions naturelles ont alors vu le jour lors de notre travail: quels sont les algorithmes qui permettent à un ordinateur
de jouer à un jeu combinatoire abstrait? Quels sont leurs points forts et points faibles? Dans cette section, nous allons
présenter deux algorithmes principaux qui aident à répondre à ces questions: la recherche arborescente \emph{Monte-Carlo} (Monte-carlo Tree Search) 
et l'algorithme \emph{MinMax} avec élagage alpha-bêta.

\subsection{La recherche arborescente Monte-Carlo}

\subsubsection{Présentation}
La recherche arborescente \emph{Monte-Carlo} ou \emph{Monte-Carlo} (MCTS) est un algorithme de recherche heuristique.
Il est principalement utilisé dans le cadre de mise en place d'intelligence artificielle pour des jeux tels que le go, mais pas uniquement.
En effet, il peut aussi être utilisé pour les moteurs de jeux des échecs comme le moteur AlphaZero de Google qui est
l'un des leaders en termes d'intelligence de jeux aux échecs. Cet algorithme peut même être implémenté dans des jeux où le hasard 
apparaît comme le poker.

\subsubsection{MCTS --- Fonctionnement succinct}
\emph{MCTS} est un algorithme qui explore l'arbre des possibles. La racine est la configuration initiale du jeu.
Chaque nœud est une configuration (une situation en jeu) et ses enfants sont les configurations suivantes.\ \emph{MCTS} conserve en mémoire 
un arbre qui correspond aux nœuds déjà explorés. Une feuille de cet arbre est soit une configuration finale (donc un des joueurs a gagné),
soit un nœud dont aucun enfant n'a encore été exploré. Dans chaque nœud, on stocke deux nombres: le nombre de simulations gagnantes, 
et le nombre total de simulations. 

À chaque itération \emph{MCTS} va déterminer la `meilleure' feuille (celle qui a le plus de potentiel) à partir de la racine.
Depuis cette feuille, il va créer grâce aux règles du jeu, une nouvelle feuille au hasard dont le but est d'atteindre une configuration finale.
Dans le cas où on trouverait une configuration gagnante pour un joueur, il va incrémenter le nombre de simulations gagnantes.
Cela vaut pour les nœuds correspondant au joueur et aussi pour les parents de la feuille qui a été créée.\@
L'algorithme répète cette opération un certain nombre de fois avant de choisir un coup, i.e: choisir le coup qui a le plus de simulations
gagnantes et le moins de simulations totales.

%exemple.\dots


\subsection{MinMax}

\subsubsection{Présentation}
L'algorithme \emph{MinMax}, comme \emph{MCTS}, est un algorithme décisionnel. Cependant, celui-ci s'applique sur des jeux à somme nulle: c'est-à-dire
un jeu où la somme des gains et des pertes de tous les joueurs est égale à 0. Ainsi, le gain de l'un constitue 
obligatoirement une perte pour l'autre. L'algorithme va utiliser cette propriété dans sa recherche pour trouver le meilleur coup possible.
En effet, l'ordinateur va passer en revue toutes les configurations possibles pour un nombre limité de coups (que l'on appellera profondeur), 
et leur assigner une valeur qui prend en compte les bénéfices pour le joueur et pour son adversaire. Le meilleur coup est alors celui qui 
minimise les pertes du joueur, tout en supposant que l'adversaire cherche au contraire à les maximiser (d'où le nom \emph{MinMax}).

\subsubsection{MinMax --- Fonctionnement succinct}
Comme le \emph{MCTS}, le \emph{MinMax} va explorer l'arbre des possibles où chaque nœud représente une configuration du jeu. 
L'algorithme va explorer toutes les possibilités et associer à chaque feuille une valeur positive ou négative. Ces feuilles sont
soit des nœuds terminaux (c'est-à-dire que l'un des deux joueurs a gagné), soit des nœuds à la profondeur de recherche maximale.
Un nœud va se voir assigner une valeur positive si la position associée favorise l'ordinateur (le joueur maximisant) 
et négative dans le cas contraire.

En ce qui concerne les nœuds feuilles non-terminaux, c'est-à-dire ceux représentant une configuration de jeu non gagnante, mais bloqués par la 
profondeur de recherche, c'est une fonction d'évaluation qui va estimer leur valeur heuristique. La qualité de cette fonction va 
déterminer ``l'intelligence'' de l'ordinateur. Cette estimation et la profondeur de recherche déterminent donc la qualité et la 
précision du résultat final du \emph{MinMax}.

Les nœuds qui ne sont pas des feuilles vont hériter d'une des valeurs de leurs enfants en fonction de s'ils sont le minimisant, ou le maximisant.
Si le rôle du nœud est de maximiser sa valeur, alors celui-ci va recevoir la plus grande valeur associée à ses fils. Sinon, son rôle est de 
minimiser sa valeur, et donc il va recevoir la plus petite valeur associée à ses fils.
Ainsi, les nœuds conduisant à un résultat favorable, comme une victoire pour le joueur maximisant, ont des scores plus
élevés que les nœuds plus favorables pour le joueur minimisant. Les valeurs des nœuds qui mènent à une victoire sont -inf ou + inf ou 0 en fonction de:
la victoire, la défaite ou l'égalité pour le joueur maximisant. 


\begin{figure}[h]
    \begin{center}
        \includegraphics[width=0.4\textwidth]{root/MinMax.jpeg}
    \end{center}
    \caption{Ici le joueur bleu cherche à maximiser ses gains avec une profondeur 4.}\label{fig:min_max}
\end{figure}


\subsubsection{Élagage alpha-bêta}
L'algorithme \emph{MinMax} effectue une exploration complète de l'arbre de recherche jusqu'à un niveau donné. L'élagage alpha-bêta permet d'optimiser 
cet algorithme sans en modifier le résultat. Pour cela, il ne réalise qu'une exploration
partielle de l'arbre. En effet, on observe qu'il n'est pas utile d'explorer les sous-arbres qui conduisent à des valeurs
qui ne participeront pas au calcul de la valeur associée à la racine de l'arbre. Dit autrement, l'élagage alpha-bêta n'évalue pas des nœuds
dont on peut penser, si la fonction d'évaluation est à peu près correcte, que leur qualité sera inférieure à celle d'un nœud déjà évalué.

\begin{figure}[h]
    \begin{center}
        \includegraphics[width=0.5\textwidth]{root/minmax_alpha_beta.png}
    \end{center}
    \caption{\emph{MinMax} avec élagage $\alpha$-$\beta$.}\label{fig:min_max_alpha_beta}
\end{figure}


Plusieurs coupures ont pu être réalisées. De gauche à droite:
\begin{enumerate}
    \item Le nœud MIN vient de mettre à jour sa valeur courante à 4. Celle-ci, qui ne peut que baisser, est déjà inférieure à $\alpha$=5, 
    la valeur actuelle du nœud MAX précédent. Celui-ci cherchant la plus grande valeur possible, ne la choisira donc de toute façon pas.
    \item Le nœud MIN vient de mettre à jour sa valeur courante à 6. Celle-ci, qui ne peut que baisser, est déjà égale à $\alpha$=6, la valeur 
    actuelle du nœud MAX précédent. Celui-ci cherchant une valeur supérieure, il ne mettra de toute façon pas à jour sa valeur que ce nœud 
    vaille 6 ou moins.
    \item Le nœud MIN vient de mettre à jour sa valeur courante à 5. Celle-ci, qui ne peut que baisser, est déjà inférieure à $\alpha$=6, la valeur 
    actuelle du nœud MAX précédent. Celui-ci cherchant la plus grande valeur possible, ne la choisira donc de toute façon pas.
\end{enumerate}

\subsubsection{Remarque}
À l'aide de l'algorithme \emph{MinMax}, il est possible de donner une preuve de l'existence d'une stratégie gagnante au jeu du Hex notamment. Nous détaillons
cette preuve en annexe.

\subsection{Autres algorithmes}
\subsubsection{Dijkstra et BFS}
Dans le projet, lorsqu'un joueur gagne au Hex, nous avons décidé de mettre en valeur son chemin gagnant.
Dans un premier temps, nous devions détecter qu'un joueur ait remporté une partie. Nous avons donc dû créer la méthode \textit{check\_winner} dont nous avons déjà parlé. Pour cela nous avons implementé l'algorithme du parcours
en largeur (\emph{BFS}). Nous avons alors interprété le tableau dans lequel les coups des joueurs sont sauvegardés comme un graphe. Le parcous en largeur
va générer un graphe couvrant depuis un bord. Ce graphe couvrant ne va passer que par les coups placés par un joueur. Si le graphe généré atteint le bord opposé,
alors un chemin relie les deux bords, et donc le joueur a gagné. Nous appellons donc cette méthode à chaque fois qu'un coup est placé. Sa complexité est $O(\lvert V \rvert + \lvert E \rvert)$.
Notons ici que nous appelons la méthode seulement sur l'arbre correspondant aux coups joués par un joueur, donc en pratique, l'appel à cet algorithme
est très rapide.

L'algorithme \emph{BFS} nous assure qu'un joueur a gagné, mais il peut exister plusieurs chemins distincts reliant les deux bords. En effet,
il peut y avoir plusieurs points de départ (points sur l'un des bords du gagnant) et points d'arrivée (points sur le bord opposé).
Afin d'afficher le plus court chemin, nous avons décidé d'intégrer une version modifiée de l'algorithme de \emph{Dijkstra}.
Nous avons opté pour une solution naïve. En effet, nous appellons l'algorithme plusieurs fois (une pour chaque point de départ possible).
Ensuite nous comparons la taille des chemins trouvés. Nous obtenons alors le plus court chemin, que l'on a choisi de mettre de jaune 
afin de faciliter sa visualisation.

À l'origine, dans le pire des cas, la complexité de l'algorithme de \emph{Dijkstra} est $O(\lvert V \rvert + \lvert E \rvert \times \log(\lvert V \rvert))$.
Cependant, notre implémentation est plus complexe, notre algorithme possède alors une complexité de $O(n \times (\lvert V \rvert+ \lvert E \rvert\times \log(\lvert V \rvert)))$, avec n la taille de notre plateau.
Notons que ce cas n'arrive en pratique jamais. En pratique le chemin le plus court est trouvé de façon instantanée.

\begin{figure}[h]
    \begin{center}
        \includegraphics[width=0.5\textwidth]{root/chemin_gagnant.png}
    \end{center}
    \caption{Ici bleu a gagné, le chemin le plus court est trouvé}\label{fig:chemin_gagnant}
\end{figure}