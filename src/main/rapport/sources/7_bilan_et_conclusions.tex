\section{Bilan et Conclusions}

% indiquer les fonctionnalités mise en oeuvre par rapport au cahier des charges
% de départ, les points ouverts et les perspectives pour le projet.
Nous avons implémenté la plupart des fonctionnalités prévues dans le cahier des charges, à l'exception de quelques
fonctionnalités mineures. Nous avons également ajouté des fonctionnalités supplémentaires qui n'étaient pas prévues
dans le cahier des charges, mais qui ont été jugées nécessaires pour améliorer la qualité du projet.

En bref, nous avons implémenté les fonctionnalités suivantes :

\begin{itemize}
    \item Possibilité pour les utilisateurs de jouer en JcJ au \emph{Hex} et à l'\emph{Awalé}
    \item Possibilité pour les utilisateurs de jouer au JcIA au \emph{Hex} et à l'\emph{Awalé}
    \item Possibilité pour les utilisateurs de regarder deux IA jouer au \emph{Hex} et à l'\emph{Awalé}
    \item Possibilité pour les utilisateurs de changer de thème de couleur pour le jeu du \emph{Hex}
    \item Possibilité pour les utilisateurs de défaire un coup dans tous les modes de jeu (sauf IAvIA), pour le \emph{Hex} et l'\emph{Awalé}
    \item Possibilité pour les utilisateurs de rejouer une partie dans tous les modes de jeu, pour le \emph{Hex} et l'\emph{Awalé}
    \item Possibilité pour les utilisateurs de choisir la taille du plateau de jeu pour le \emph{Hex}
    \item Possibilité pour les utilisateurs de choisir son camp (en JcIA) pour le \emph{Hex} et l'\emph{Awalé}
    \item Possibilité pour les développeurs de tester, déployer et maintenir facilement l'application
    \item Possibilité pour les développeurs de rajouter facilement de nouvelles fonctionnalités
    \item Possibilité pour les développeurs de rajouter facilement de nouveaux jeux
    \item Possibilité de changer le thème visuel pour le hex (en appuyant sur la touche H)
\end{itemize}

Si nous avions continué le projet plus longtemps, nous aurions implémenté les fonctionnalités suivantes :

\begin{itemize}
    \item Possibilité de jouer en ligne contre un autre joueur
    \item Possibilité de choisir un niveau de difficulté pour l'IA
    \item Implémentation d'autres algorithmes de jeu comme \emph{Monte-Carlo} par exemple
    \item Mise en place d'une base de données pour enregistrer des scores
    \item Tableau des "high scores"
    \item Implémentation d'autres jeux combinatoires abstraits
    \item Ajout de sons (musiques et effets sonores)
\end{itemize}